
\documentclass[12pt]{article}
%% preamble.tex

% enclose any latex in \comment{} to suppress it
\newcommand{\comment}[1]{}
\newcommand{\citeNP}[1]{\cite{#1}}

\usepackage{graphicx}
\usepackage{multicol}
\usepackage{times}
\usepackage{floatflt}
\graphicspath{{./}{figures/}}

% NO psfig -- pdflatex does not support it ...
% \usepackage{psfig}
% \psfigurepath{./:figures/}
% \DeclareGraphicsRule{.eps.gz}{eps}{.eps.bb}{'gunzip -c #1}

\newcommand{\upline}{\vspace*{-\baselineskip}}
\newcommand{\up}{\vspace*{-6pt}}
\newcommand{\downline}{\vspace*{\baselineskip}}
\newcommand{\sep}{~~~~~~~~~~}

% this hardcodes the bib name, which we don't want to do.
% \renewcommand{\thebibliography}[1]{\section*{References Cited}
% \addcontentsline{toc}{section}{References Cited}\list
%  {[\arabic{enumi}]}{\settowidth\labelwidth{[#1]}\leftmargin\labelwidth
%  \advance\leftmargin\labelsep
%  \usecounter{enumi}}
%  \def\newblock{\hskip .11em plus .33em minus -.07em}
%  \sloppy\clubpenalty4000\widowpenalty4000
%  \sfcode`\.=1000\relax}

% fix this to specify width and height, and solve for the margins
% margins.tex

% import calc
\usepackage{calc}

%
\newlength{\myrightmargin}
\newlength{\myleftmargin}
\newlength{\mytopmargin}
\newlength{\mybottommargin}

% Change these settings to change the margins
\setlength{\myrightmargin}{1.0in}
\setlength{\myleftmargin}{1.0in}
\setlength{\mytopmargin}{0.75in}     
\setlength{\mybottommargin}{0.75in} 
\setlength{\oddsidemargin}{0.0in}   % extra room on inside side

%%% use margin settings to set width variables
\setlength{\evensidemargin}{0 in}
\setlength{\marginparsep}{0 in}
\setlength{\marginparwidth}{0 in}
\setlength{\hoffset}{\myleftmargin - 1.0in}
\setlength{\textwidth}
  {8.5in -\myleftmargin -\myrightmargin -\oddsidemargin}

%%% use margin settings to set height variables
\setlength{\voffset}{\mytopmargin -1.0in}
\setlength{\topmargin}{0 in}
\setlength{\headheight}{12 pt}
\setlength{\headsep}{20 pt}
\setlength{\footskip}{36 pt}
\setlength{\textheight}
  {11.0in-\mytopmargin-\mybottommargin-\headheight-\headsep-\footskip}

% \oddsidemargin 0.2cm
% \evensidemargin 0cm
% \textwidth 16.0cm
% \topmargin -1.25cm
% \textheight 22.94cm

% remove parindent, squeeze grafs
\setlength{\parindent}{0in}
\setlength{\parskip}{1ex}

\def\nibf#1{\noindent\textbf{#1}}

\usepackage{multirow}
\usepackage{amsmath}
\usepackage{fancyhdr}
\usepackage[left=1in,top=1.0in,right=1in,headsep=0.25in,footskip=0.3in,bottom=0.7in]{geometry}
%\usepackage[left=1in,top=1.2in,right=1in,footskip=0.3in,bottom=0.7in,showframe]{geometry}
%\usepackage[left=1in,top=1in,right=1in,bottom=1in,nohead]{geometry}
\usepackage{graphicx}
\renewcommand{\arraystretch}{1} % spacing between table rows
\usepackage[]{caption}
\setlength{\abovecaptionskip}{0pt}
\setlength{\belowcaptionskip}{-5pt}
\setlength{\intextsep}{10pt plus 2pt minus 2pt}
\usepackage[normalem]{ulem}
\newcounter{Labcounter}
\newcounter{Taskcounter}
\numberwithin{equation}{section}
\fancyhf{}
\pagestyle{fancy}
\fancypagestyle{plain}{ %
% \fancyhf{} % remove everything
\renewcommand{\headrulewidth}{0pt} % remove lines as well
%\renewcommand{\footrulewidth}{0pt}
%\cfoot{Page \thepage~of \pageref{LastPage}}}
\cfoot{\thepage}
}


\lhead{\textit{Nanospheres...}}
%\chead{\Large \textbf{Paul W.~Leu } \vspace{0.3em}}
%\chead{\Large \textbf{NSF Biographical Sketch: Paul W.~Leu} \vspace{0.3em}}
\rhead{\textit{Leu}}
\newcommand{\blue}[1]{\textcolor{blue}{#1}} %for displaying red texts

\newcommand{\vectornorm}[1]{\left|\left|#1\right|\right|}
%\usepackage[top=2.5cm, bottom=2.5cm, left=2.5cm, right=2.5cm]{geometry}
\usepackage[normalem]{ulem}
\newenvironment{packed_enum}{
\begin{enumerate}
  \setlength{\topsep}{0pt}
  \setlength{\partopsep}{0pt}
  \setlength{\itemsep}{1pt}
  \setlength{\parskip}{0pt}
  \setlength{\parsep}{0pt}
}{\end{enumerate}}

%\usepackage[small]{caption}
\usepackage[draft]{pdfcomment}
\usepackage{wrapfig}
\usepackage{hyperref}
\usepackage{paralist}
\usepackage{amsmath}
\usepackage{amssymb}
\usepackage{amsfonts}
\usepackage{textcomp}
\usepackage{subfig}
\usepackage{framed}
\usepackage{setspace}
\usepackage{here}
\usepackage[numbers, square, comma, sort&compress]{natbib}

\usepackage[compact]{titlesec}
\titlespacing{\section}{0pt}{0ex}{0pt}
\titlespacing{\subsection}{0pt}{0pt}{0pt}
\usepackage{xcolor}

\usepackage[]{caption}
\setlength{\abovecaptionskip}{0pt}
\setlength{\belowcaptionskip}{-5pt}
\setlength{\intextsep}{10pt plus 2pt minus 2pt}


\usepackage{float}
\floatstyle{plaintop}
\newfloat{program}{thp}{lop}
\floatname{program}{Table}

\newfloat{wrapprogram}{thp}{lop}
\floatname{wrapprogram}{Table}

%\setlength{\intextsep}{10pt plus 2pt minus 2pt}
% bold face: highlight keywords, or big ideas
% italics: inconspicuous stressing of key points
% underline: hypothesis; avoid

\usepackage{bm}

\date{\today }

\author{
Paul W. Leu\\
University of Pittsburgh\\
Pittsburgh, PA}


\newcommand{\red}[1]{\textcolor{red}{#1}} %for displaying red texts; answers to short text


%\def\myTitle{CAREER: Transforming Solar Energy Harvesting through Nanophotonic Light Trapping}
%\def\myTitle{CAREER: Characterizing Enhanced Absorption and Carrier Collection Mechanisms in Silicon and Zinc Oxide Nanocone-based Solar Cells}
%Upper bound
%   100 hours/year
%    40 hours
%    
%\title{Ultimate Limits of Silicon Nanostructures for Photon Management\\
%\title{Determining Silicon/Metal Nanostructures that Approach the Wave-Optics Light Trapping Limit by Data Mining of Electrodynamic Simulations}
%\title{CAREER: Characterizing Enhanced Absorption and Carrier Collection Mechanisms in Silicon and Zinc Oxide Nanocone-based Solar Cells}
%\title{Predicting Wave-Optics Light Trapping in Silicon and Metal Nanostructure by Data Mining of Electrodynamic Simulations}
%Keywords: ultra thin, light trapping, wave-optics light trapping, coherent light trapping, optimization framework, data mining}

%\title{Modeling and Manufacturing of Three Dimensional Silicon Nanostructures for Photon Management}

\begin{document}

\tableofcontents

\section{Surface Plasmon Polaritons at Single Interface}

Surface plasmon polaritons only exist for TM polarization.  
\begin{equation}
\boxed{\beta =  k_0  \sqrt{ \frac{\epsilon_1 \epsilon_2}{\epsilon_1 + \epsilon_2} }}
\end{equation}
This expression is valid for both real and complex $\epsilon_1$, i.e. for conductors without
and with attenuation.

For a square lattice, \cite{Thio:99}
\begin{equation}
\bm{\beta}_{sp} = \bm{k_x} + m \bm{G_x} + n \bm{G_y}
\end{equation}
where 
\begin{equation}
\bm{k_x} = k_0  \sin \theta = \frac{2 \pi}{\lambda} \sin \theta 
\end{equation}
is the component of the wave vector of the incident light that lies in the plane ($\theta = 0$ for normal incidence).  
$\bm{G_x}$ and $\bm{G_y}$ are the reciprocal lattice vectors, and $m$ and $n$ are integers.  

For a square lattice at normal incidence
\begin{equation}
| \bm{\beta}_{sp} | = \sqrt{m^2 + n^2} \frac{2 \pi}{a_0} = k_0  \sqrt{ \frac{\epsilon_1 \epsilon_2}{\epsilon_1 + \epsilon_2} }
\end{equation}

For a triangular lattice, 
$\bm{a_1} = a \bm{x}$ and $\bm{G_2} = \frac{a}{2} \bm{x} + \frac{\sqrt{3} a}{2} \bm{y}$
The reciprocal lattice vectors are 
$\bm{G_1} = \frac{2 \pi}{\sqrt{3} a} \left ( \sqrt{3} \bm{k_x} - \bm{k_y} \right )$ and 
$\bm{G_2} = \frac{4 \pi}{\sqrt{3} a} \bm{k_y} $
Thus, for a square lattice at normal incidence
\begin{equation}
| \bm{\beta}_{sp} | = \frac{4 \sqrt{3}}{3 a} \sqrt{m^2 - mn + n^2}
\end{equation}
\red{Note that this is different from \cite{Thio:99}, which has 
$| \bm{\beta}_{sp} | = \frac{4 \sqrt{3}}{3 a} \sqrt{m^2 + mn + n^2}$.
If use $\bm{a_1} = a \bm{x}$ and $\bm{G_2} = - \frac{a}{2} \bm{x} + \frac{\sqrt{3} a}{2} \bm{y}$, 
then get 
$\bm{G_1} = \frac{2 \pi}{\sqrt{3} a} \left ( \sqrt{3} \bm{k_x} + \bm{k_y} \right )$ and 
$\bm{G_2} = \frac{4 \pi}{\sqrt{3} a} \bm{k_y} $ and get same answer
\begin{equation}
| \bm{\beta}_{sp} | = \frac{4 \sqrt{3}}{3 a} \sqrt{m^2 + mn + n^2}
\end{equation}
Both should be okay as you get the same integers for $m^2 + mn + n^2$ and $m^2 - mn + n^2$.  
}

The traveling SPPs are damped with an energy attenuation length (also called the propgation length) 
\begin{equation*}
\begin{aligned}
L &= (2 \textrm{ Im} [ \beta ] )^{-1} \\
&= \left ( 2 \textrm{ Im} \left [ k_0 \sqrt{ \frac{\epsilon_1 \epsilon_2}{\epsilon_1 + \epsilon_2}} \right ] \right )^{-1} \\
&= \left ( 2 \textrm{ Im} \left [ \frac{2 \pi}{\lambda} \sqrt{ \frac{\epsilon_1 \epsilon_2}{\epsilon_1 + \epsilon_2}} \right ] \right )^{-1}
\end{aligned}
\end{equation*}




\begin{equation}
\end{equation}


%\blue{Some issues with some modes missing.}

\bibliographystyle{IEEEtran}
\bibliography{../../../help/BibFiles/Plasmonics,../../../help/BibFiles/Nanosphere}


\end{document}
