
\documentclass[12pt]{article}
%% preamble.tex

% enclose any latex in \comment{} to suppress it
\newcommand{\comment}[1]{}
\newcommand{\citeNP}[1]{\cite{#1}}

\usepackage{graphicx}
\usepackage{multicol}
\usepackage{times}
\usepackage{floatflt}
\graphicspath{{./}{figures/}}

% NO psfig -- pdflatex does not support it ...
% \usepackage{psfig}
% \psfigurepath{./:figures/}
% \DeclareGraphicsRule{.eps.gz}{eps}{.eps.bb}{'gunzip -c #1}

\newcommand{\upline}{\vspace*{-\baselineskip}}
\newcommand{\up}{\vspace*{-6pt}}
\newcommand{\downline}{\vspace*{\baselineskip}}
\newcommand{\sep}{~~~~~~~~~~}

% this hardcodes the bib name, which we don't want to do.
% \renewcommand{\thebibliography}[1]{\section*{References Cited}
% \addcontentsline{toc}{section}{References Cited}\list
%  {[\arabic{enumi}]}{\settowidth\labelwidth{[#1]}\leftmargin\labelwidth
%  \advance\leftmargin\labelsep
%  \usecounter{enumi}}
%  \def\newblock{\hskip .11em plus .33em minus -.07em}
%  \sloppy\clubpenalty4000\widowpenalty4000
%  \sfcode`\.=1000\relax}

% fix this to specify width and height, and solve for the margins
% margins.tex

% import calc
\usepackage{calc}

%
\newlength{\myrightmargin}
\newlength{\myleftmargin}
\newlength{\mytopmargin}
\newlength{\mybottommargin}

% Change these settings to change the margins
\setlength{\myrightmargin}{1.0in}
\setlength{\myleftmargin}{1.0in}
\setlength{\mytopmargin}{0.75in}     
\setlength{\mybottommargin}{0.75in} 
\setlength{\oddsidemargin}{0.0in}   % extra room on inside side

%%% use margin settings to set width variables
\setlength{\evensidemargin}{0 in}
\setlength{\marginparsep}{0 in}
\setlength{\marginparwidth}{0 in}
\setlength{\hoffset}{\myleftmargin - 1.0in}
\setlength{\textwidth}
  {8.5in -\myleftmargin -\myrightmargin -\oddsidemargin}

%%% use margin settings to set height variables
\setlength{\voffset}{\mytopmargin -1.0in}
\setlength{\topmargin}{0 in}
\setlength{\headheight}{12 pt}
\setlength{\headsep}{20 pt}
\setlength{\footskip}{36 pt}
\setlength{\textheight}
  {11.0in-\mytopmargin-\mybottommargin-\headheight-\headsep-\footskip}

% \oddsidemargin 0.2cm
% \evensidemargin 0cm
% \textwidth 16.0cm
% \topmargin -1.25cm
% \textheight 22.94cm

% remove parindent, squeeze grafs
\setlength{\parindent}{0in}
\setlength{\parskip}{1ex}

\def\nibf#1{\noindent\textbf{#1}}

\usepackage{multirow}
\usepackage{amsmath}
\usepackage{fancyhdr}
\usepackage[left=1in,top=1.0in,right=1in,headsep=0.25in,footskip=0.3in,bottom=0.7in]{geometry}
%\usepackage[left=1in,top=1.2in,right=1in,footskip=0.3in,bottom=0.7in,showframe]{geometry}
%\usepackage[left=1in,top=1in,right=1in,bottom=1in,nohead]{geometry}
\usepackage{graphicx}
\renewcommand{\arraystretch}{1} % spacing between table rows
\usepackage[]{caption}
\setlength{\abovecaptionskip}{0pt}
\setlength{\belowcaptionskip}{-5pt}
\setlength{\intextsep}{10pt plus 2pt minus 2pt}
\usepackage[normalem]{ulem}
\newcounter{Labcounter}
\newcounter{Taskcounter}
\numberwithin{equation}{section}
\fancyhf{}
\pagestyle{fancy}
\fancypagestyle{plain}{ %
% \fancyhf{} % remove everything
\renewcommand{\headrulewidth}{0pt} % remove lines as well
%\renewcommand{\footrulewidth}{0pt}
%\cfoot{Page \thepage~of \pageref{LastPage}}}
\cfoot{\thepage}
}


\lhead{\textit{Nanospheres...}}
%\chead{\Large \textbf{Paul W.~Leu } \vspace{0.3em}}
%\chead{\Large \textbf{NSF Biographical Sketch: Paul W.~Leu} \vspace{0.3em}}
\rhead{\textit{Leu}}
\newcommand{\blue}[1]{\textcolor{blue}{#1}} %for displaying red texts

\newcommand{\vectornorm}[1]{\left|\left|#1\right|\right|}
%\usepackage[top=2.5cm, bottom=2.5cm, left=2.5cm, right=2.5cm]{geometry}
\usepackage[normalem]{ulem}
\newenvironment{packed_enum}{
\begin{enumerate}
  \setlength{\topsep}{0pt}
  \setlength{\partopsep}{0pt}
  \setlength{\itemsep}{1pt}
  \setlength{\parskip}{0pt}
  \setlength{\parsep}{0pt}
}{\end{enumerate}}

%\usepackage[small]{caption}
\usepackage[draft]{pdfcomment}
\usepackage{wrapfig}
\usepackage{hyperref}
\usepackage{paralist}
\usepackage{amsmath}
\usepackage{amssymb}
\usepackage{amsfonts}
\usepackage{textcomp}
\usepackage{subfig}
\usepackage{framed}
\usepackage{setspace}
\usepackage{here}
\usepackage[numbers, square, comma, sort&compress]{natbib}

\usepackage[compact]{titlesec}
\titlespacing{\section}{0pt}{0ex}{0pt}
\titlespacing{\subsection}{0pt}{0pt}{0pt}
\usepackage{xcolor}

\usepackage[]{caption}
\setlength{\abovecaptionskip}{0pt}
\setlength{\belowcaptionskip}{-5pt}
\setlength{\intextsep}{10pt plus 2pt minus 2pt}


\usepackage{float}
\floatstyle{plaintop}
\newfloat{program}{thp}{lop}
\floatname{program}{Table}

\newfloat{wrapprogram}{thp}{lop}
\floatname{wrapprogram}{Table}

%\setlength{\intextsep}{10pt plus 2pt minus 2pt}
% bold face: highlight keywords, or big ideas
% italics: inconspicuous stressing of key points
% underline: hypothesis; avoid

\usepackage{bm}

\date{\today }

\author{
Paul W. Leu\\
University of Pittsburgh\\
Pittsburgh, PA}


%\def\myTitle{CAREER: Transforming Solar Energy Harvesting through Nanophotonic Light Trapping}
%\def\myTitle{CAREER: Characterizing Enhanced Absorption and Carrier Collection Mechanisms in Silicon and Zinc Oxide Nanocone-based Solar Cells}
%Upper bound
%   100 hours/year
%    40 hours
%    
%\title{Ultimate Limits of Silicon Nanostructures for Photon Management\\
%\title{Determining Silicon/Metal Nanostructures that Approach the Wave-Optics Light Trapping Limit by Data Mining of Electrodynamic Simulations}
%\title{CAREER: Characterizing Enhanced Absorption and Carrier Collection Mechanisms in Silicon and Zinc Oxide Nanocone-based Solar Cells}
%\title{Predicting Wave-Optics Light Trapping in Silicon and Metal Nanostructure by Data Mining of Electrodynamic Simulations}
%Keywords: ultra thin, light trapping, wave-optics light trapping, coherent light trapping, optimization framework, data mining}

%\title{Modeling and Manufacturing of Three Dimensional Silicon Nanostructures for Photon Management}

\begin{document}

\tableofcontents

\section{Nanowires}

For nanowires, the leaky resonance modes are according to Fountaine \cite{Fountaine:14}
\begin{equation}
\begin{aligned}
\pm \left ( \frac{1}{k_{cyl}^2 - k_{air}^2} \right ) ^2 \left ( \frac{k_z m}{k_0 a} \right )^2 = \left ( \frac{\epsilon_{cyl}}{k_{cyl}} \frac{J_m' (k_{cyl} a )}{J_m (k_{cyl} a )} - \frac{1}{k_{air}} \frac{H_m' (k_{air} a)}{H_m (k_{air} a )} \right ) 
\\ \times 
\left ( \frac{1}{k_{cyl}} \frac{J_m' (k_{cyl} a )}{J_m (k_{cyl} a )} - \frac{1}{k_{air}} \frac{H_m' (k_{air} a)}{H_m (k_{air} a )} \right ) 
\end{aligned}
\end{equation}
$k_{cyl}$ is the transverse component of the wavevector inside the cylinder.  
\begin{equation}
k_{cyl} = \left [ \left ( \frac{n_{cyl} \omega}{c} \right )^2 - \beta^2 \right ]^{1/2}
\end{equation}

$k_{air} = \left [ \left ( \frac{\omega}{c} \right )^2 -  \beta^2 \right ]^{1/2}$



According to Linyou's dissertation, 
\begin{equation}
\kappa^2 = k_0^2 n^2 - \beta^2
\end{equation}
\begin{equation}
\gamma^2 = k_0^2 - \beta^2
\end{equation}
$\kappa$ and $\gamma$ are the wave vectors in the transverse direction inside and outside of the cylindrical structure.  
If $\gamma^2 < 0$ and $\kappa^2 > 0$, then it is a guided mode.  
If $\gamma^2 > 0$, then it is a leaky mode.  

\begin{equation}
\left ( \frac{1}{\kappa^2} - \frac{1}{\gamma^2} \right ) ^2 \left ( \frac{\beta m}{a} \right )^2
= k_0^2 \left ( n^2 \frac{J_m' (\kappa a )}{\kappa J_m (\kappa a )} - n_0^2 \frac{H_m' (\gamma a)}{\gamma H_m (\gamma a )}   \right )
\left ( \frac{J_m' (\kappa a )}{\kappa J_m (\kappa a )} - \frac{H_m' (\gamma a)}{\gamma H_m (\gamma a )}   \right )
\end{equation}
This is the same as above, where
$\kappa = k_{cyl}$ and $\gamma = k_{air}$.
$\beta = k_z$. 
$n^2 = \epsilon_{cyl}$. 



%
%
%\textbf{Leaky mode} versus \textbf{Guided mode}.
%
%
%
%According to Snyder, for guided modes, 
%\begin{equation}
%\begin{aligned}
%\left ( \frac{\nu \beta}{k n_{co}} \right )^2 \left ( \frac{V}{U W} \right )^4= \left ( \frac{J_\nu' (U )}{U J_\nu ( U )} +  \frac{K_\nu' (W )}{W K_\nu ( W )} \right ) 
%\\ \times 
%\left ( \frac{J_\nu' (U )}{U J_\nu ( U )} + \frac{n_{cl}^2}{n_{co}^2} \frac{K_\nu' (W )}{W K_\nu ( W )} \right )
%\end{aligned}
%\end{equation}
%where the core paramater $U = \rho (k^2 n_{co}^2 - \beta^2 )^{1/2}$ and 
%the cladding parameter 
%$W = \rho (\beta^2 - k^2 n_{cl}^2 )^{1/2}$.
%$V^2 = U^2 + W^2$.  
%For radiation modes, 
%$Q = \rho (k^2 n_{cl}^2 - \beta^2 )^{1/2} = i W$.  
%$Q = a \gamma$.
%$V^2 = U^2 - Q^2$.  
%$W = - Q i = - a \gamma i$.  
%
%
%Substituting $\nu = m$ and 
%$k = k_{0}$.  
%$\rho = a$.  
%$U = a \kappa$ and $W = - i a \gamma $
%
%
%\begin{equation}
%\begin{aligned}
%\left ( \frac{\beta m}{k n_{co}} \right )^2 \left ( \frac{V}{U W} \right )^4= \left ( \frac{J_m' (U )}{U J_m ( U )} +  \frac{K_m' (W )}{W K_m ( W )} \right ) 
%\\ \times 
%\left ( \frac{J_m' (U )}{U J_m ( U )} + \frac{n_{cl}^2}{n_{co}^2} \frac{K_m' (W )}{W K_m ( W )} \right )
%\end{aligned}
%\end{equation}


Inside a subwavelength hole, the propagating modes are 
\cite{Catrysse:08}
\begin{equation}
\begin{aligned}
\left ( \frac{1}{k_{cyl}^2 - k_1^2} \right ) ^2 \left ( \frac{k_z m}{k_0 a} \right )^2 = \left ( \frac{\epsilon_{cyl}}{k_{cyl}} \frac{J_m' (k_{cyl} a )}{J_m (k_{cyl} a )} - \frac{\epsilon_1}{k_{1}} \frac{H_m' (k_1 a)}{H_m (k_1 a )} \right ) 
\\ \times 
\left ( \frac{1}{k_{cyl}} \frac{J_m' (k_{cyl} a )}{J_m (k_{cyl} a )} - \frac{1}{k_{air}} \frac{H_m' (k_1 a)}{H_m (k_1 a )} \right ) 
\end{aligned}
\end{equation}





\subsection{Modal Cutoff}

The lower limit $\beta_j = k n_{cl}$ of permissble values of the bound mode propagation constant is called 
modal cutoff.  Modal cutoff, or just cutoff is defined by 
\begin{equation}
U_j = V; W_j = 0.
\end{equation}
Below cutoff, these modes propagate with loss and are the leaky modes.  
\blue{Leaky modes are also referred to as quasiguided modes.}



According to \cite{Snyder:83}, 
leaky modes are bound modes below cutoff.  The modal parameters are now complex.  

The leaky-mode propagation constant satisfies, 
\begin{equation}
0 \leq \beta^{R} < k n_{cl}
\end{equation}
and 
\begin{equation}
U^r > V
\end{equation}.

The cladding parameter is $Q = \left [ \beta^2 - \left ( \frac{n_1 \omega}{c} \right )^2   \right ]^{1/2} = i W$.

Just below cutoff, $W$ is almost pure imaginary.  
$Q$ is almost real.  

Imaginary part is indicative of the radiative loss of the mode.  This transitions to zero as the mode transitions from leaky to guided.  





%\blue{Some issues with some modes missing.}

\bibliographystyle{IEEEtran}
\bibliography{../../../help/BibFiles/Photodetector,../../../help/BibFiles/SiliconPV,../../../help/BibFiles/Nanosphere,../../../help/BibFiles/Plasmonics,../../../help/BibFiles/PDOS}


\end{document}
